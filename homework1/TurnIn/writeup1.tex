\documentclass[draftclsnofoot,onecolumn,letterpaper,10pt,compsoc]{IEEEtran}

%%%%%%%%%%%%%%%%%%%%%%%%%%%%%%%%%%%%%%%%%%%%%%%%%%%%%%%%%%%%%%%%%%%%%%%%%%%%%%%%
% Preamble
%% Note: Files in this directory are included in parent directory docs.
%% This means you need to source local files as `extras/file_name` instead of `file_name`.
%%%%%%%%%%%%%%%%%%%%%%%%%%%%%%%%%%%%%%%%%%%%%%%%%%%%%%%%%%%%%%%%%%%%%%%%%%%%%%%%

\newcommand{\subparagraph}{}

\usepackage[T1]{fontenc}
\usepackage[backend=bibtex]{biblatex}
\usepackage[english]{babel}
\usepackage[margin=0.75in]{geometry}
\usepackage[pdf]{pstricks}
\usepackage[utf8]{inputenc}
\usepackage{alltt}                                           
\usepackage{amsmath}                                         
\usepackage{amssymb}                                         
\usepackage{amsthm}                                          
\usepackage{blindtext}
\usepackage{calc}
\usepackage{caption}
\usepackage{color}
\usepackage{csquotes}
\usepackage{enumitem}
\usepackage{float}
\usepackage{geometry}
\usepackage{graphicx}
\usepackage{hyperref}
\usepackage{listings}
\usepackage{longtable}
\usepackage{pst-gantt}
\usepackage{pst-uml}
\usepackage{subcaption}
\usepackage{titlesec}
\usepackage{titling}
\usepackage{url}

\newcommand{\subtitle}[1]{%
  \posttitle{%
    \par\end{center}
    \begin{center}\large#1\end{center}
    \vskip0.5em}%
}

\geometry{textheight=8.5in, textwidth=6in}

\graphicspath{ {figures/} } 

\newcommand{\cred}[1]{{\color{red}#1}}
\newcommand{\cblue}[1]{{\color{blue}#1}}
\newcommand{\inlinecode}[1]{\texttt{#1}}

\definecolor{mygreen}{rgb}{0,0.6,0}
\definecolor{mygray}{rgb}{0.5,0.5,0.5}
\definecolor{mymauve}{rgb}{0.58,0,0.82}

\lstset{ %
  backgroundcolor=\color{white},   % choose the background color; you must add \usepackage{color} or \usepackage{xcolor}
  basicstyle=\footnotesize\ttfamily,        % the size of the fonts that are used for the code
  breakatwhitespace=false,         % sets if automatic breaks should only happen at whitespace
  breaklines=true,                 % sets automatic line breaking
  captionpos=b,                    % sets the caption-position to bottom
  commentstyle=\color{mygreen},    % comment style
  deletekeywords={...},            % if you want to delete keywords from the given language
  escapeinside={\%*}{*)},          % if you want to add LaTeX within your code
  extendedchars=true,              % lets you use non-ASCII characters; for 8-bits encodings only, does not work with UTF-8
}

\newcommand{\namesigdate}[2][5cm]{%
  \begin{tabular}{@{}p{#1}@{}}
    #2 \\[2\normalbaselineskip] \hrule \\[0pt]
    {\small \textit{Signature}} \\[2\normalbaselineskip] \hrule \\[0pt]
    {\small \textit{Date}}
  \end{tabular}
}

\newcommand{\subsubsubsection}[1]{%
  \begin{flushleft}
  \paragraph{#1}
  \end{flushleft}
}

\newpsstyle{Important}{fillstyle=solid, fillcolor=red}
\newpsstyle{NotImportant}{fillstyle=vlines }


\usepackage[T1]{fontenc}
\usepackage{geometry}
\geometry{textwidth=10cm}
\sffamily
\usepackage{listings}

\definecolor{povcodered}{rgb}{0.75,0.25,0.25}

\title{
	Project One Writeup \\
    {
    	% Document Sub-title
    	\LARGE Spring 2017 
    }
}
\author{
  \IEEEauthorblockN{Shuai Peng (pengs),}
  \IEEEauthorblockN{Anya Lehman (lehmana),}
  \IEEEauthorblockN{Andrew Bowers (bowerand),}
}

%%%%%%%%%%%%%%%%%%%%%%%%%%%%%%%%%
% Document
%%%%%%%%%%%%%%%%%%%%%%%%%%%%%%%%%
\begin{document}

%%%%%%%%%%%%%%%%%%%%%%%%%%%%%%%%%
% Title Page
%%%%%%%%%%%%%%%%%%%%%%%%%%%%%%%%%
\maketitle
\begin{abstract}
  This is our write up for the project one \textit{Getting Acquainted}. This project included setting up our environment and an exercise on concurrency. Listed below is the process we took to set out the environment as well as the process we took to complete the concurrency exercise. Note: All text in red is code.
\end{abstract}

\clearpage

%%%%%%%%%%%%%%%%%%%%%%%%%%%%%%%%%
% Command Log
% > briefly recaps the steps we took along the path of project one
%%%%%%%%%%%%%%%%%%%%%%%%%%%%%%%%%
\section{Command Log}
\begin{enumerate}
    \item To start we logged on to the os-class \textcolor{povcodered}{ssh username@os-class.oregonstate.edu}
 \item Then we used cd to get to the correct folder in scratch/spring2017 \textcolor{povcodered}{cd /scratch/spring2017}
 \item Then we made a group folder for us all to work in \textcolor{povcodered}{mkdir 11-04}
 \item Next we struggled to make said folder accessible to all of our gorup members by changing the permitions on it so not just the group member that cr    eated the directory could work in it \textcolor{povcodered}{chmod 777 11-04}
 \item Then we called git clone to download the project from the GitHub account and we checked to make sure we got all the correct files \textcolor{povcodered}{git clone git://git.yoctoproject.org/linux-yocto-3.14}
 \item Then we switched to the tag we needed by using cd again and going into the directory that was cloned into our folder \textcolor{povcodered}{cd linux-yocto-3.14}
 \item Following this we checked out the v3.14.26 \textcolor{povcodered}{git checkout v3.14.26}
 \item Next came configuring the environment which we did by calling \textcolor{povcodered}{source /scratch/opt/environment-setup-i586-poky-linux}
    \item Then we made a kernel instance for our group
 \item Then we coppied in the files that lets us configure \textcolor{povcodered}{cp /scratch/spring2017/files/config-3.14.26-yocto-qemu .config}
 \item Then we ran \textcolor{povcodered}{make menuconfig}
    \item A window popped up
 \item In this window we pressed \textcolor{povcodered}{/} and typed \textcolor{povcodered}{LOCALVERSION}
 \item Next we pressed \textcolor{povcodered}{1} and edited the value to be \textcolor{povcodered}{-11-04-hw1} to make that the name of the kernel
 \item Then we built our kernel with four threads by running \textcolor{povcodered}{-j4}
 \item Then we ran \textcolor{povcodered}{cd ..} followed by \textcolor{povcodered}{gdb}
 \item Our next step was to move onto a different laptop and called \textcolor{povcodered}{source /scratch/opt/environment-setup-i586-poky-linux } again
 \item Then we made a copy for the starting kernel and the drive file located in the scratch directory by calling \textcolor{povcodered}{/scratch/spring2017/files/core-    image-lsb-sdk-qemux86.ext3}
 \item Then we tried running the starting kernel \textcolor{povcodered}{qemu-system-i386 -gdb tcp::5604 -S -nographic -kernel bzImage-qemux86.bin -drive file=core-image    -lsb-sdk-qemux86.ext3,if=virtio -enable-kvm -net none -usb -localtime --no-reboot --append "root=/dev/vda rw console=ttyS0 debug"}
 \item Since we previously ran the qemu in debug mode, we used gdb to control it so, back on the origonal computer, we connected the qemu by running \textcolor{povcodered}{target remote :5604}
 \item Then we rebooted the VM \textcolor{povcodered}{reboot}
 \item Then we tried running the kernel instance we had created \textcolor{povcodered}{linux-yocto-3.14/arch/x86/boot/}
 \item Then we ran \textcolor{povcodered}{qemu-system-i386 -gdb tcp::5601 -S -nographic -kernel linux-yocto-3.14/arch/x86/boot/bzImage  -drive file=core-image-lsb-sdk-q    emux86.ext3,if=virtio -enable-kvm -net none -usb -localtime --no-reboot --append "root=/dev/vda rw console=ttyS0 debug"}
 \item Finally we rebooted the vm and used \textcolor{povcodered}{q} to quit
\end{enumerate}

\section{Write Up of Concurrency Solution}
We noticed that the assignment required functionality for a consumer, producer, and random number generator operations. The best solution we thought of was to create functions for each of these operations, then use main to generate our threads and handle the operations of the program. Our main function intitialzes our values, creates the threads, then loops through calling the different functions as necessary. Once the producer function is called, it calls the random generator function to create the numbers to be stored in the struct, which are stored in the buffer once the struct is created. We used mutex locking to protect data from other threads. Theproducer then waits it's wait period, which is also randomly generated. In a similar fashion, when the consumer fucntion is called it will pull a struct from the buffer, wait for the given amount of time, given by the second number in the struct, then print out the first number. As noted earlier, main loops through creating threads and calling the functions. 

\section{Explaition of the Flags}
\begin{enumerate}
\item \textcolor{povcodered}{-gdb} makes the device wait for a connection to the gdb debug tool.
\item \textcolor{povcodered}{-nongraphic} means make it a non graphical interface that works like the command line.
\item \textcolor{povcodered}{ -kernal} means to use the following kernel.
\item \textcolor{povcodered}{-drive} means to use the following driver.
\item \textcolor{povcodered}{-enablekvm} enables KVM in full virtualization support.
\item \textcolor{povcodered}{-net} means to use the follwing network.
\item \textcolor{povcodered}{-usb} enables the USB driver.
\item \textcolor{povcodered}{-localtime} its setting the time for the local machine.
\item \textcolor{povcodered}{--no-reboot} exits instead of rebooting. 
\item \textcolor{povcodered}{--append} uses the command line as the kernel command line.
\item \textcolor{povcodered}{-s} changes it to strict mode which makes it so that it fails on different image size or sector allocation.
\end{enumerate}

\section{Questions Regarding Concurrency}
\begin{enumerate}

   \item \textit{What do you think the main point of this assignment is} The main point of this assignment is to get us to learn more about conncerency and to build up our ability to understand and think in parallel. Parallel computing is the simultaneous use of multiple resources to solve a computational problem. This is important because compared to serial computing, parallel computing is much better for tasks such as modeling, simulating, and working on more complex, real world problems. The conceptual problem of this assingmnt is not as challenging as understuanding how to implement it. The secondary point of this assignment is to get us familiar with the techniques we need for the rest of this class. 

   \item \textit{How did you personally approach the problem? Design decisions, algorithm, etc.} To start this problem, we first worked out the pseudo code with pen and paper.  We designed it so that the Consumer is a function and the Producer is a function. Then we also created a function that generated different random numbers based on cases\cite{1} \cite{3} (case for the 3-7 second waiting period, the 2-9 second waiting period, and the random number that the consumer prints out). We went about this by using a struct to hold the values that the Producer creates and the Consumer are taking. The first number in each struct will be the randomly gemerated number for the consumer to print and the second will be the time the consumer should wait. This struct is generated in our main function. We also created a function that checked to see if RdRand works on the current system.\cite{2}   

   \item \textit{How did you ensure your solution was correct? Testing details, for instance.} To ensure that our solution was correct we used printf to print out all the step by step details that are not normally shown so that we knew exactly what was going on behind the scene.

   \item \textit{What did you learn?} This assignment taught us how to properly create a program that uses parallel programming. We re-learned a lot of old material about writing in c and utilizing pthreads. We also learned the control you can gain with inline assembly. This project also helped us understand how we work as a group.  

\end{enumerate}

\section{Version Control Log}
\begin{center}
\begin{tabular}{ |c|c|c| }
   File Version & Group Member(s) & What Was Done \\
   \hline \hline
   V1 & Andrew & Created the concerency file to work from \\
   \hline
   V2 & Andrew & Developed the functions\\
   \hline
   V2 & Shuai & Implemented the functions\\
   \hline
   V3 & Andrew & Developed the structs and the consumer and producer\\
   \hline
   v4 & Shaui & Adapted the mt19937\\
   \hline
   v5 & Shaui & Debugged the file\\
   \hline
   v6 & Anya & Created the rdrand function\\
   \hline
   v7 & All & Compiled it all together\\
   \hline
\end{tabular}
\end{center}

\section{Work Log}
\begin{center}
\begin{tabular}{ |c|c|c|c|c|c| }
   Date & Group Member(s) & Start Time & End Time & Total Time Worked & Accomplished \\ 
   \hline \hline
   April 11th & All & 11:00am & 12:00pm & 1 hour & Met our group members and\\
    &  &  &  &  & exchanged contact information. \\
    &  &  &  &  & Set up our working environment \\
    &  &  &  &  & including creating our kernel \\
    &  &  &  &  & and making the group folder to\\
    &  &  &  &  & work in. Made sure all members \\
    &  &  &  &  & of the group could access the \\
    &  &  &  &  & group folder and the group \\
    &  &  &  &  & GitHub account. \\
   \hline
   April 18th & All & 11:00am & 12:00pm & 1 hour & Set up the folder to contain \\
    &  &  &  &  & the work we will be doing\\
    &  &  &  &  & in this project. Started on\\
    &  &  &  &  & the writeup by setting it\\
    &  &  &  &  & up and completing the write\\
    &  &  &  &  & up for the log of commands.\\
    &  &  &  &  & Got the github to work.\\
    &  &  &  &  & We also read through the\\
    &  &  &  &  & concerency assignment. \\  
   \hline
   April 19th & All & 2:30pm & 8:30pm & 5 hours & Begain the concerency assignment.\\
    &  &  &  &  & Created the psudeo code for the\\
    &  &  &  &  & exersise and discused how we \\
    &  &  &  &  & wanted to go about completing \\
    &  &  &  &  & the exersise. Created the c file\\
    &  &  &  &  & and made our struct and all of \\
    &  &  &  &  & our functions. Also worked on\\
    &  &  &  &  & the write up to create the work\\
    &  &  &  &  & log and begain to asnwer the\\
    &  &  &  &  & questions regarding the \\
    &  &  &  &  & concerency exercise. \\
   \hline
   April 20th & All & 5:00pm & 10:30pm & 5.5 hours & Almost finished the concerency assignment.\\
    &  &  &  &  & Debugged the project and made\\
    &  &  &  &  & it so that the loop went for \\
    &  &  &  &  & our desired length. Also made\\
    &  &  &  &  & the two random generating stlyes\\
   \hline
   April 21th & Shuai & 12:00pm & 1:00pm & 1 hour & Fixed the compile error\\
   \hline
   April 21th & Andrew, Anya & 12:00pm & 7:45pm & 7.75 hours & Finished the concerency assignment.\\
    &  &  &  &  & Redid most of the concerency.\\
    &  &  &  &  & Edited the makefile..\\
    &  &  &  &  & Also wrapped up the writeup and\\
    &  &  &  &  & turned it all in.\\
   \hline
\end{tabular}
\end{center}

\section{Citations}
\begin{thebibliography}{9}
   \bibitem{Mersenne}
      Takuji Nishimura and Makoto Matsumoto.
      \textit{Mersenne Twister with improved initialization}.
      MT, 1997-2002.
   \bibitem{POSIX}
      Blaise Barney.
      \textit{POSIX Threads Programming}.
      Lawrence Livermore National Laboratory, 2017.
   \bibitem{intel}
      Intel Corporation.
      \textit{Intel 64 and IA-32 Architectures Software Developer's Manual: Basic Architecture'}.
      Order Number: 325462-060US, Intel technologies, 2016.
   \bibitem{stack}
      Dirk Eddelbuettel.
      \textit{Catch trl-C in C}.
      Stack Overflow, 2016.
   \bibitem{IBM}
      IBM Knowledge Center.
      \textit{pthread kill -- Send Signal to Thread}.
      IMB

   \end{thebibliography}
\end{document}
